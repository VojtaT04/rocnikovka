\section {Využití pulzní oxymetrie}
Jak již bylo zmíněno v úvodu, přístroje operující na principu pulzní oxymetrie jsou v našem životě již velmi běžné. To, co bylo před 80 lety naprostou technologickou novinkou, která byla používána jen pro vojenské účely, je v současnosti běžnou součástí chytrých náramků a hodinek, které podle jednoho výzkumu \citep{wearables} využíval v roce 2019 více než jeden z pěti Američanů. Ačkoliv nejsou zatím podobná data, jež by měřila změnu v tomto trendu od počátku epidemie Covidu, která prokazatelně zvýšila zájem lidí o vlastní zdraví a životní styl, dá se očekávat, že tato čísla se výrazně zvýšila, a to i u starší generace, jež podle výše zmíněného výzkumu využívala tato zařízení v srovnání s mladší a střední generací o třetinu méně.
\par Zároveň je však tato metoda ve většině ohledů stále nepřekonaná a využívá se tedy stále velmi běžně ve zdravotnictví - ať už na jednotkách intenzivní péče, na operačních sálech nebo třeba pro novorozence. V poslední době jsou zároveň rozšiřovány možnosti této metody tak, aby dokázala měřit například přítomnost různých škodlivých látek, které by jinak byly vyhodnoceny jako normální okysličená krev.
\subsection {Sport}
Velmi častou metodou používanou při tréninku u vrcholových sportovců je cvičení v horších kyslíkových podmínkách, což znamená, že sportovci trénují za nižší koncentrace vdechovaného kyslíku ($FiO_2$). Nicméně, jak naznačují například \cite{fio2}, výsledná hodnota $SpO_2$ je ve vztahu k $FiO_2$ velmi vysoce individuální. To je způsobeno tím, že koncentrace kyslíku je velmi úzce spojena s jeho parciálním tlakem, který, jak již bylo ukázáno na obrázku \ref{fig:PO2}, nemá s $SpO_2$ lineární vztah. Pokud tedy chceme upravit přísun kyslíku, tak aby došlo ke konkrétnímu posunu v $SpO_2$, musíme si uvědomit, že je velmi vysoký rozdíl mezi minimální a maximální očekávanou výslednou hodnotou $SpO_2$. 
\par To znamená, že pro optimalizaci benefitů pro sportovce není vhodné nastavovat vnější prostředí na konkrétní hodnotu. Namísto toho je vhodné trénovat v adaptivním prostředí, které používá dat z pulzního oxymetru sportovce k přizpůsobení koncentrace kyslíku. Jde tedy o stejný princip jako u originálního Millikanova oxymetru, s tím rozdílem, že pro sportovce je naopak žádoucí kyslík držet na nižší hladině. \citep{sportuse}
\subsubsection {Pulz}
Vzhledem k tomu, že při funkci pulzního oxymetru je měřen i tep, je pulzní oxymetr velmi vhodný pro i pro amatérské a občasné sportovce. Různé tepové zátěže (v poměru k maximální tepové frekvenci) odpovídají různým typům cvičení, která mají různé cíle, tak jak je uvedeno v tabulce \ref{tab:MTF}. Pulzní oxymetry v chytrých náramcích a hodinkách, jež jsou zdaleka nejčastějšími pulzními oxymetry pro sportovce, mají tedy i možnost ukazovat typ cvičení a díky tomu i různé rady a informace pro uživatele.¨
\par Samotný výpočet maximální tepové frekvence je pro takové zařízení však těžší. Vzorce pro tento výpočet obecně nejsou uznávány jako dostatečně přesné vzhledem k různorodosti každého člověka. Ačkoliv byly v minulosti studie, které se snažily vzorce upřesňovat na základě různých parametrů, nepanuje shoda na žádném vzorci ani na proměnných důležitých pro výpočet. Příkladem může být studie, která jako jediný parametr uvádí věk a od něj odvozuje jeden z nejpoužívanějších vzorců pro odhad ($HR_{Max}=211-0,64\times\text{věk}$). I tato studie však dochází k závěru, že "Předpovídání maximální tepové frekvence podle věku může být pro různé skupiny velmi pohodlné, avšak je potřeba brát v úvahu standardní odchylku 10,8 tepů za minutu." \citep{maxHR}
\begin{table}[h]
    \centering
    \begin{tabular}{p{0.25\linewidth} | p{0.1\linewidth} | p{0.30\linewidth} | p{0.24\linewidth}}
        \textbf{Typ cvičení}            & \textbf{\% Max TF} & \textbf{Cíl cvičení}                                              & \textbf{Fyzický pocit}              \\ \hline
        Lehké aerobní                   & 55-70                    & Kardiovaskulární kondice, redukce hmotnosti                       & Uvolněný, možnost mluvit bez obtíží \\
        Středně těžké aerobní           & 70-80                    & Kardiovaskulární kondice, zvyšování výkonnosti                    & Teplo, pocení, mírně zrychlený dech \\
        Hranice aerobního a anaerobního & 80-85                    & Kardiovaskulární kondice, zvyšování odolnosti                     & Teplo, pocení, namáhavé dýchání     \\
        Na kyslíkový dluh               & 85-90                    & Zvyšování výkonnosti, spalování kyslíku ve svalech, srdeční zátěž & Nadměrné pocení, lapání po dechu    \\
        Čistě anaerobní                 & 90-100                   & Zvykání těla na práci bez spalování kyslíku, maximální výkon      & Vysoký stres, nadměrné pocení, pocit dušení
    \end{tabular}
    \caption[Typy cvičení podle maximální tepové frekvence]{Jednou z důležitých sledovaných hodnot pro sportovce je tepová zátěž, která odpovídá různým typům cvičení. \citep{tep}}%doplnit citaci
    \label{tab:MTF}
\end{table}
\subsection {Zdravotnictví}
Jak již bylo zmíněno, oxymetry mají své využití napříč medicínou - od neonatologie až po chirurgii. Při monitorování pacientova stavu je saturace kyslíku, vedle tepové frekvence, krevního tlaku a EKG, jedním z hlavních fyziologických ukazatelů pacientova stavu. Všechny tyto hodnoty jsou monitorovány neinvazivním způsobem a vyhodnocovány na jednom monitoru, což umožňuje lékařům  dělat rychlé a přesné závěry o pacientově stavu. Vzhledem k tomu, že tato měření jsou průběžná, mají ještě jeden velmi důležitý význam - upozorní lékaře v případě, že by některé hodnoty byly mimo normu a ohrožovaly pacienta.
\par Tyto možnosti jsou důvodem, proč se pulzní oxymetry, spolu s dalšími již zmíněnými přístroji, využívají na jednotkách intenzivní péče, operačních a porodních sálech, při péči o novorozence nebo například v sanitních vozech. Jak uvádí \cite{monitoring}, existují minimálně 3 typy pacientů, kteří potřebují průběžné měření. Prvním typem jsou pacienti, kteří mají dočasně poškozené tělní mechanismy starající se o udržování konkrétních hodnot. Příkladem může být celková anestezie, jež má ve většině zemí světa přísné regulace na potřebné monitorovací vybavení. Druhou kategorií jsou pak pacienti u nichž lze vzhledem k jejich stavu očekávat náhlou změnu stavu. Příkladem této kategorie mohou být novorozenci, jejich matky či pacienti po operaci srdce. Třetí kategorií jsou pak pacienti v kritickém stavu. To znamená například pacienty, jež utrpěli vážné zranění, při němž došlo k vážnému poranění vnitřních orgánů nebo protržení plíce. Patří sem i pacienti s vážnými infekcemi či dokonce septickým šokem.
\par Ačkoliv se nejedná o hlavní využívanou metodu pro tyto případy, je potřeba zmínit i kyslíkovou terapii, při níž je pacientům ve stavu hypoxie podáván kyslík tak, aby nedošlo k poškození tkání. \cite{O2terapie} uvádí, že s kyslíkovou terapií by se mělo přestat přibližně při 90\% saturaci kyslíku.