\section{Demonstrace funkčnosti oxymetru}
Kromě výše uvedeného popisu vývoje produktu si tato práce klade za cíl demonstrovat funkčnost vyvinutého produktu měřením malého vzorku lidí během fyzické aktivity. Cílem tohoto měření není získat nové poznatky, ale demonstrovat reálná naměřená data doplněná o jejich popis.
\par Je velice pravděpodobné, že tato naměřená data nebudou mít vysokou výpovědní hodnotu, což není na škodu vzhledem k tomu, že hlavním úkolem této části je seznámit případné uživatele či zájemce o projekt s reálnými výsledky. Tito uživatelé si následně mohou v duchu \emph{open-source} vylepšit jak software, tak hardware pro své potřeby tak, aby byli schopni dosáhnout lepších výsledků při měření.
\subsection{Cíle výzkumu}
Cílem prováděného výzkumu bude měření tepu a $SpO_2$ během cvičení u malého množství dospívajících ve věku 15-18 let. Tato věková skupina byla zvolena mimo jiné i pro svoji v průměru velmi dobrou fyzickou kondici, která je nutným předpokladem pro intenzivní cvičení, během něhož jsou data měřena. Konkrétní výzkumná otázka, kterou se bude tento výzkum zabývat zní „\emph{Jaký vliv má krátkodobé intenzivní cvičení na tepovou frekvenci a saturaci kyslíku v krvi u dospívajících osob?}“
\subsection{Hypotézy}
Na základě výzkumné otázky byla stanoveny následující hypotézy pro výzkum:
\par\textbf{Hypotéza 1:} během intenzivního cvičení lze na oxymetru pozorovat zvyšující se tepovou frekvenci ($BPM$).
\par\textbf{Hypotéza 2:} během intenzivního cvičení lze na oxymetru pozorovat snižující se saturaci kyslíku v krvi ($SpO_2$).
\subsection{Metodika výzkumu}
Se svojí účastí ve výzkumu souhlasilo 11 jedinců obou pohlaví ve zkoumané věkové kategorii (15-18 let). Všichni účastníci byli poučeni o zásadách výzkumu, s nimiž souhlasili (příloha \ref{appn:Agreement}), správném používání pulzního oxymetru a byli seznámeni s následujícím postupem měření:
\par Každý účastník si nasadí na svůj prst oxymetr, který bude následně spuštěn. Účastníci poté v maximální rychlosti uběhnou deset krátkých úseků po schodech nahoru a dolů. Jejich data budou následně z oxymetru uložena do počítače.
\par Cílem tohoto postupu je vystavit účastníky intenzivní krátkodobé fyzické zátěži tak, aby došlo k obtížím s dýcháním. Díky tomu lze očekávat změnu $SpO_2$ a $BPM$.
\par V poslední řadě bude provedeno i kontrolní měření, při němž bude jedna osoba sedět po delší dobu v klidu pro demonstraci funkčnosti oxymetru v klidovém režimu.
