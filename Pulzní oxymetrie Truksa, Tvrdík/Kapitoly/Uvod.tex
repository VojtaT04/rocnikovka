\newpage
\part{Úvod}
Pulzní oxymetrie se dostává do života kolem nás, aniž bychom o tom věděli. Ve skutečnosti většina lidí ani přibližně neví, co by si pod tímto slovním spojením měla představit. Přesto je to právě pulzní oxymetrie, která dnes a denně pomáhá lékařům po celém světě s monitorováním pacientů a zachraňováním jejich životů. Nikdy nebyla potřeba zjišťovat rychle a přesně kyslík v krvi tak důležitá, jako právě v současné době.
\par Kvůli pandemii Covidu-19, nemoci způsobené virem SARS-CoV-2, se stalo, že se ve vlnách přeplňovaly nemocnice po celém světě a na jednotkách intenzivní péče docházela místa. Vzhledem k tomu, že SARS-CoV-2, jak už část názvu napovídá (SARS = Severe Acute Respiratory Syndrome), způsobuje potíže s dýcháním, je u pacientů, převážně ve vážnějším stavu, naprosto nezbytné monitorovat saturaci kyslíku v krvi.
\par Kromě tohoto medicínského využití, kde je potřeba velice vysoká přesnost, jsou však ještě další, kde na přesnosti přístroje nezávisí životy pacientů. Běžní uživatelé totiž nevyžadují naprostou přesnost a ani jednorázová chyba přístroje je nemůže ohrozit - ať už využívají oxymetr při cvičení pro hlídání zátěže, nebo například zjišťují, jestli nemají ve spánku problém s dýcháním. Pro tyto účely se na trhu v současnosti vyskytují dva typy produktů.
\par Prvním z nich jsou chytré hodinky a náramky: když si člověk vybaví chytré hodinky, je velká šance, že si je představí se svítícím zeleným světýlkem v části, která přiléhá na ruku. To je právě jedna z diod, která má za úkol zjišťovat pulz a saturaci kyslíku v krvi. Nevýhodou těchto produktů je však jejich cena, spojená se všemi funkcemi, které s sebou chytré hodinky nesou. Nejsou tedy například vhodné, pokud se člověk rozhodne je využít pro sledování spánku svého staršího příbuzného.
\par Druhým typem produktu jsou samostatné pulzní oxymetry, které jsou, zjednodušeně řečeno, tak přesné, jak jen může běžný uživatel potřebovat. I jejich cena je relativně přijatelná - v současnosti se pohybuje v řádu vyšších stovek korun. I tyto oxymetry jsou však pro většinu uživatelů přesnější, než kdy využijí, což znamená, že se cena dá ještě snížit použitím domácky vyrobeného oxymetru.
\par V teoretické části této práce detailně vysvětlíme, jak pulzní oxymetrie funguje, jakých principů využívá a jak se zpracovávají naměřená data, a rozebereme její využití v reálném životě - od profesionálního vybavení, až po nejobyčejnější nepravidelné kontrolování pulzu. V praktické části následně provedeme čtenáře postupem při vyvíjení „domácího“ oxymetru, od volby platformy, přes 3D tisk, až po software. Návod na stavbu je včetně všech tipů zveřejněn pod open-sourcovou licencí na webu, aby kdokoliv mohl s běžně dostupnými součástmi a za velice dobrou cenu postavit svůj vlastní oxymetr a následně si ho upravit pro své specifické potřeby.
\par Tento pulzní oxymetr bude také v rámci práce vyzkoušen v rámci výzkumu týkajícího se obsahu kyslíku v krvi v závislosti na fyzické námaze. Cílem výzkumu nebude přinést nové poznatky a nová data, ale demonstrovat funkčnost výsledného produktu a způsob zacházení s ním.